%
%  $Author: ienne $
%  $Date: 1995/09/15 15:20:59 $
%  $Revision: 1.4 $
%

% \documentclass[10pt,journal,cspaper,compsoc]{IEEEtran}   %%%tc version
\documentclass[10pt, conference]{IEEEtran}
%\documentclass[conference,compsoc]{IEEEtran}
%\documentclass[10pt, conference]{IEEEtran}
%\documentclass[times, 10pt,onecolumn]{article}
\usepackage{amsmath, amssymb, enumerate}

%%%%%%%%%%%%%%%% page control%%%%%%%%%%%%%%%%%
%\usepackage[margin=0.75in]{geometry}

%\linespread{0.991}  %%%%%%%%%%%%%%%%%%%%%%%%%%%%%%%%% this is really useful
%\usepackage{cite}
\usepackage{fancybox}
\usepackage{amsfonts}
%\usepackage{algorithm}
%\usepackage[noend]{algorithmic}
\usepackage[usenames]{color}
%\usepackage{colortbl}
%\usepackage[ figure, boxed, vlined]{algorithm2e}
%\usepackage[linesnumbered,vlined]{algorithm2e}
%\usepackage[lined,boxed]{algorithm2e}
\usepackage{listings}

\usepackage[linesnumbered,vlined]{algorithm2e}
\usepackage{graphicx}
\usepackage{times}
\usepackage{psfrag}
\usepackage{subfigure}
\usepackage{caption}
%\usepackage{subcaption}
\usepackage{multirow}
%\usepackage{setspace}
%\usepackage{listings}
\usepackage{epsfig}
%\usepackage{epstopdf}
%\usepackage[font=small,labelfont=bf]{caption}
\usepackage{url}

\usepackage{color}
\def\fixme#1{\typeout{FIXED in page \thepage : {#1}}
%\bgroup \color{red}{} \egroup}
\bgroup \color{red}{[FIXME: {#1}]} \egroup}


%\usepackage[pdftex]{hyperref}
\usepackage{rotating,tabularx}

\interfootnotelinepenalty=10000

%% Define a new 'leo' style for the package that will use a smaller font.
\makeatletter
\def\url@leostyle{%
  \@ifundefined{selectfont}{\def\UrlFont{\sf}}{\def\UrlFont{\small\ttfamily}}}
\makeatother

%\documentstyle[times,art10,twocolumn,latex8]{article}

%-------------------------------------------------------------------------
% take the % away on next line to produce the final camera-ready version
\pagestyle{plain}
%\thispagestyle{empty}
%\pagestyle{empty}

\newtheorem{theorem}{Theorem}
\newtheorem{lemma}[theorem]{Lemma}

%% remaining budget share, used in task stall section.
\newcommand{\bottomrule}{\hline}
\newcommand{\toprule}{\hline}
\newcommand{\midrule}{\hline}
%-------------------------------------------------------------------------
\begin{document}

\title{DeepPicar:​ ​A​ ​Low-cost​ ​Deep​ ​Neural​ ​Network-based​ ​Autonomous​ ​Car}
\author{Author One, Author Two\\
\{author1,author2\}@ku.edu\\
University of Kansas, USA\\ 
}

\maketitle
\thispagestyle{empty}
\begin{abstract}

Abstract goes here.

\end{abstract}

%-------------------------------------------------------------------------

\section{Introduction}
Autonomous vehicles have been a topic of increasing interest in many disciplines in recent years. Due to the nature of autonomous vehicles, it is necessary that all real-time operations are successfully performed prior to their deadlines. This requires that the AV platform be capable of completing all necessary computations in a timely manner, while maintaining a high level of accuracy. Consequently, the cost of creating a platform capable of self-driving can be relatively high, especially in the current cost-sensitive climate of the automotive industry \cite{}. The overall price for an autonomous vehicle platform currently acts as a bottleneck, especially from an academic standpoint, where cost is an important factor. For many researchers and students, the charge associated with autonomous vehicle research is a significant barrier. These same parties may also be deterred from participating in autonomous vehicle research due to the potential for their developed platforms to break in the experimental process. The idea of building one relatively expensive platform may not be too daunting for some, but the notion of potentially making multiple identical platforms in case of accidents may be overwhelming. Toward this end, we explore the possibility of developing a low-cost system that still employs state-of-the-art AI technologies and is capable of executing real-time autonomous vehicle operations.
	
In this paper, we present DeepPicar, or Picar for short, a low-cost autonomous car platform that can be utilized in research and education. We follow the same methodology found in other platforms, such as NVIDIA’s end-to-end vision based control \cite{}, but implement it on a smaller scale. The Picar is comprised of a Raspberry Pi 3 Model B, a camera and a low cost RC car. By creating and using the Picar platform, we seek to accomplish the following three goals. First, the significant reduction of the overall cost required would make autonomous driving research more accessible to interested individuals/parties. Furthermore, it would remove the concerns of creating replacement platforms as each part, or the entire system, could be replaced relatively inexpensively. Second, the utilization of a more cost-efficient platform would allow for a better understanding of the performance requirements necessary for efficiently operating autonomous vehicles. Finally, we strive to achieve the ability to compare the real-time performance of multiple embedded computing platforms.

We display the efficacy of the Picar by training and testing the DeepTesla DNN \cite{} in a custom-made environment. We find that the Picar is capable of consistently accomplishing all real-time operations required for autonomous driving in a 50 millisecond frame.

The remaining sections of the paper are as follows: Section II gives an overview of the platform, including the high-level system and the methods used for training and inference. Section III discusses the ways/methodologies in which training data was collected. Section IV outlines the network architecture. This is followed by the real-time DNN inferencing in Section V. Section VI reviews the online/offline training done with the platform, with an evaluation given in Section VII. Section VIII gives a discussion of related works, and the paper finishes with conclusions in Section IX.

\section{Background}
Cite a paper \cite{barroso2009datacenter}.

Cite multiple papers \cite{banga99resourcecontainers,barroso2009datacenter}

\section{Platform Overview}
...
\section{Data Collection}
...
\section{Network Architecture}
...
\section{Real-Time DNN Inferencing}
...
\section{Online/Offline Training}
...
\section{Evaluation}
...
\section{Related Work}
...
\section{Conclusion}
%-------------------------------------------------------------------------

\bibliographystyle{plain}
\bibliography{reference}
\end{document}

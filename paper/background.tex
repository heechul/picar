\section{Background} \label{sec:background}

In this section, we provide background on the application of deep
learning in robotics, particularly autonomous vehicles. 

%% \subsection{Deep Convolutional Neural Network}
%% \fixme{CNN background}

\subsection{End-to-End Deep Learning for Autonomous Vehicles}

%% - explosion of AI
%% - in particular, application of DNN in perception and control of robotics systems.
%% - end-to-end control is a promising technique.
%%   levine's publications?
%% - examples: nvidia's DAVE-II prototype, forest navigating drone
%% challenge problem: computing at low cost?

To solve the problem of autonomous driving, a standard approach has
been decomposing the problem into multiple sub-problems,
such as lane marking detection, path planning, and low-level
control, which together form a processing pipeline~\cite{Bojarski2016}.
Recently, researchers have begun exploring another approach that dramatically
simplifies the standard control pipeline by applying deep neural
networks to directly produce control outputs from sensor
inputs~\cite{Levine2016}. Figure~\ref{fig:end-to-end-control}
shows the differences between the two approaches.

\begin{figure}[h]
  \centering
  \includegraphics[width=.5\textwidth]{figs/endtoend_redrawn}
  \caption{Standard robotics control vs. DNN based end-to-end
    control. Adopted from ~\cite{Levine2017cs294}.}
  \label{fig:end-to-end-control}
\end{figure}

The use of neural networks for end-to-end control of autonomous
cars was first demonstrated in the late 1980s~\cite{Pomerleau1989},
using a small 3-layer fully connected neural network; and subsequently
in a DARPA Autonomous Vehicle (DAVE) project in early
2000s~\cite{LeCun:04}, using a 6 layer convolutional neural network
(CNN); and most recently in NVIDIA's DAVE-2
project~\cite{Bojarski2016}, using a 9 layer CNN. In all of these projects,
the neural network models take raw image pixels as input and directly
produce steering control commands, bypassing all intermediary steps and
hand-written rules used in the conventional robotics control approach.  
NVIDIA's latest effort reports that their trained CNN
autonomously controls their modified cars on public roads without human
intervention~\cite{Bojarski2016}.

Using deep neural networks involves two distinct
phases~\cite{NVIDIA2015}. The first
phase is \emph{training}, during which the weights of the network are
incrementally updated by backpropagating errors it sees from the
training examples. Once the network is trained---i.e., the weights of
the network minimize errors in the training examples---the next phase
is \emph{inferencing}, during which unseen data is fed to the network
as input to produce predicted output (e.g., predicted image
classification). In general, the training phase is more computationally
intensive and requires high throughput, which is generally not
available on embedded platforms. The inferencing phase, on the
other hand, is relatively less computationally intensive and latency becomes
as important, if not moreso, as computational throughput, because many
use cases have strict real-time requirements.

%% (e.g., search query latency)

%% \cite{Levine2016}: ``In this paper, we aim to answer
%% the following question: does training the perception and control
%% systems jointly end-toend 
%% provide better performance than training each component separately?''

%% \cite{Bojarski2016} nvidia paper
%% ``We trained a convolutional neural network (CNN) to map raw pixels from
%% a sin- gle front-facing camera directly to steering commands.''

%% ``Compared to explicit decomposition of the problem, such as lane
%% marking detec- tion, path planning, and control, our end-to-end system
%% optimizes all processing steps simultaneously. We''

%% UPenn's f1/10 BOM: $3,628.37
%% http://f1tenth.org/
%% http://selfdrivingcars.mit.edu/
%% http://fast.scripts.mit.edu/racecar/
%% https://github.com/mit-racecar
%% https://mit-racecar.github.io/

\subsection{Embedded Computing Platforms for Real-Time Inferencing}
%% \fixme{we can consider moving this section to a dedicated related work
%%   section at the end.}
Real-time embedded systems, such as an autonomous vehicle, present
unique challenges for deep learning, as the computing platforms of such
systems must satisfy two often conflicting goals:
%% Recent successes in AI, including NVIDIA's DAVE-2 showing, are due
%% in large part to the increased computing performance,
%% which afforded researchers to train and use ever deeper neural networks with
%% high accuracy.
%% For practical applications, the computer platform in an
%% autonomous vehicle must satisfy two often conflicting goals:
(1) The platform must provide 
enough computing capacity for real-time processing of computationally
expensive AI workloads (deep neural networks); and
(2) The platform must also satisfy various
constraints such as cost, size, weight, and power consumption limits~\cite{Otterness2017}.

Accelerating AI workloads, especially inferencing
operations, has received a lot of attention from academia and industry
in recent years as applications of deep learning are broadening to include
areas of real-time embedded systems such as autonomous vehicles. These
efforts include the development of various heterogeneous architecture-based 
system-on-a-chip (SoCs) that may include multiple cores, GPU,
DSP, FPGA, and neural network optimized ASIC hardware~\cite{Jouppi2017}.
Consolidating multiple tasks on SoCs with a lot of shared hardware
resources while guaranteeing real-time performance is also an active
research area, which is orthogonal to improving raw
performance. Consolidation is necessary for efficiency, but unmanaged 
interference can nullify the benefits of consolidation~\cite{Kim2016}.
%% For these reasons, finding a good computing platform is a
%% non-trivial task, one that requires a deep understanding of the
%% workloads and the hardware platform being utilized.

The \emph{primary objectives of this study} are (1) to understand the
necessary computing performance to realize deep neural network based
robotic systems, (2) to understand the characteristics of the
computing platform to support such workloads, and (3) to evaluate the
significance of contention in shared hardware resources and existing
mitigation techniques to address the contention problem.

To achieve these goals, we implement a low-cost autonomous car platform
as a case-study and systematically conduct experiments, which we will 
describe in the subsequent sections.

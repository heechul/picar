%
%  $Author: ienne $
%  $Date: 1995/09/15 15:20:59 $
%  $Revision: 1.4 $
%

% \documentclass[10pt,journal,cspaper,compsoc]{IEEEtran}   %%%tc version
\documentclass[10pt, conference]{IEEEtran}
%\documentclass[conference,compsoc]{IEEEtran}
%\documentclass[10pt, conference]{IEEEtran}
%\documentclass[times, 10pt,onecolumn]{article}
\usepackage{amsmath, amssymb, enumerate}

%%%%%%%%%%%%%%%% page control%%%%%%%%%%%%%%%%%
%\usepackage[margin=0.75in]{geometry}

%\linespread{0.991}  %%%%%%%%%%%%%%%%%%%%%%%%%%%%%%%%% this is really useful
%\usepackage{cite}
\usepackage{fancybox}
\usepackage{amsfonts}
%\usepackage{algorithm}
%\usepackage[noend]{algorithmic}
\usepackage[usenames]{color}
%\usepackage{colortbl}
%\usepackage[ figure, boxed, vlined]{algorithm2e}
%\usepackage[linesnumbered,vlined]{algorithm2e}
%\usepackage[lined,boxed]{algorithm2e}
\usepackage{listings}

\usepackage[linesnumbered,vlined]{algorithm2e}
\usepackage{graphicx}
\usepackage{times}
\usepackage{psfrag}
\usepackage{subfigure}
\usepackage{caption}
%\usepackage{subcaption}
\usepackage{multirow}
%\usepackage{setspace}
%\usepackage{listings}
\usepackage{epsfig}
%\usepackage{epstopdf}
%\usepackage[font=small,labelfont=bf]{caption}
\usepackage{url}

\usepackage{color}
\def\fixme#1{\typeout{FIXED in page \thepage : {#1}}
%\bgroup \color{red}{} \egroup}
\bgroup \color{red}{[FIXME: {#1}]} \egroup}


%\usepackage[pdftex]{hyperref}
\usepackage{rotating,tabularx}

\interfootnotelinepenalty=10000

%% Define a new 'leo' style for the package that will use a smaller font.
\makeatletter
\def\url@leostyle{%
  \@ifundefined{selectfont}{\def\UrlFont{\sf}}{\def\UrlFont{\small\ttfamily}}}
\makeatother

%\documentstyle[times,art10,twocolumn,latex8]{article}

%-------------------------------------------------------------------------
% take the % away on next line to produce the final camera-ready version
\pagestyle{plain}
%\thispagestyle{empty}
%\pagestyle{empty}

\newtheorem{theorem}{Theorem}
\newtheorem{lemma}[theorem]{Lemma}

%% remaining budget share, used in task stall section.
\newcommand{\bottomrule}{\hline}
\newcommand{\toprule}{\hline}
\newcommand{\midrule}{\hline}
%-------------------------------------------------------------------------
\begin{document}

\title{DeepPicar:​ ​A​ ​Low-cost​ ​Deep​ ​Neural​ ​Network-based​ ​Autonomous​ ​Car}
\author{Author One, Author Two\\
\{author1,author2\}@ku.edu\\
University of Kansas, USA\\ 
}

\maketitle
\thispagestyle{empty}
\begin{abstract}

Abstract goes here.

\end{abstract}

%-------------------------------------------------------------------------

\section{Introduction} \label{sec:intro}

% broad context:
% - advance in ai sparked interests in the robotics application, such as
%   self-driving cars.
% - in particular, deep neural network models are increasingly used
%   for perception and control of a vehicle. say. AI workloads.
%
%
Autonomous cars have been a topic of increasing interest in recent
years as many companies are actively developing related hardware
and software technologies toward fully autonomous driving capability with
no human intervention. Deep neural networks (DNNs) have been
successfully applied in various perception and control tasks in
recent years.  They are important workloads for autonomous vehicles
as well. For example, Tesla Model S was known to use a specialized
chip (MobileEye EyeQ), which used a deep neural network for vision-based
real-time obstacle detection and avoidance. More recently, researchers
are investigating DNN based end-to-end control of
cars~\cite{Bojarski2016} and other robots. It is expected that more
DNN based Artificial Intelligence workloads may be used in future
autonomous vehicles.

% big problem
Executing these AI workloads on an embedded computing platform 
poses several additional challenges. First, many AI workloads in vehicles 
are computationally demanding and have strict real-time requirements. 
For example, latency in a vision-based object
detection task may be directly linked to safety of the vehicle. This
requires a high computing capacity as well as the means to guaranteeing
the timings. On the other hand, the computing hardware platform must
also satisfy cost, size, weight, and power constraints, which require a
highly efficient computing platform. These two conflicting
requirements  complicate the platform selection process as observed in
~\cite{Otterness2017}.

%% For example, while today's self-driving car
%% prototype equip more \$100,000
%% of computers and sensors~\cite{juliussen2014emerging}, a study
%% found that aveage consumers are willing to pay much less amount of
%% extra cost for a self-driving capability~\cite{Daziano2017}.
%% https://qz.com/924212/what-it-really-costs-to-turn-a-car-into-a-self-driving-vehicle/

% related work and remaining problems
To understand what kind of computing hardware is needed for AI
workloads, we need a testbed and realistic workloads. While using a real 
car-based testbed would be most ideal, it is not only highly expensive, but also
poses serious safety concerns that hinder development and exploration.
Therefore, there is a strong need for safer and less costly
testbeds. There are already several relatively inexpensive RC-car
based testbeds, such as MIT's 
RaceCar~\cite{shin2017project} and UPenn's F1$/$10 racecar~\cite{upennf1tenth}.
However, these RC-car testbeds still cost more than \$3,000, requiring
considerable investment.

% our goals
Instead, we want to build a low cost testbed that still employs the
state-of-the art AI technologies. Specifically, we focus on a end-to-end
deep learning based real-time control system,
which was developed for a real self-driving car, NVIDIA
DAVE-2~\cite{Bojarski2016}, and use the same methodology on a
smaller and less costly setup. In developing the testbed, our
goals are (1) to analyze real-time issues in DNN based end-to-end
control; and (2) to evaluate real-time performance of contemporary embedded
platforms for such workload.

% DeepPicar introduction
In this paper, we present DeepPicar, a low-cost autonomous car
platform for research. From a hardware perspective,
DeepPicar is comprised of a Raspberry Pi 3 Model B quad-core
computer, a web camera and a RC car, all of which are affordable
components (less than \$100 in total).
The DeepPicar, however, employs state-of-the-art AI
technologies, including a vision-based end-to-end control system that
utilizes a deep convolutional neural network (CNN).
The network receives an image frame from a single forward
looking camera as input and generates a predicted steering angle
value as output at each control period in \emph{real-time}.
The network has 8 layers, about 27 million connections
and 250 thousand parameters (weights).
The network architecture is identical to that of NVIDIA's DAVE-2
self-driving car~\cite{Bojarski2016}, which uses a much more powerful
computer (Drive PX computer~\cite{drivepx}) than a Raspberry Pi~3.
We chose to use a Pi 3 not only because it is affordable, but also because it is representative
of today's mainstream low-end embedded multicore platforms found in
smartphones and other embedded devices.

%% Other than the difference in scale (RC car vs. real car), the only other
%% differences between the two systems---from the computing
%% perspective---are that our system is implemented in
%% TensorFlow~\cite{abadi2016tensorflow} and runs on a Raspberry Pi 3
%% whereas NVIDIA's DAVE-2 systems is implemented in Torch
%% 7~\cite{collobert2011torch7} and runs on a Drive PX computer (NVIDIA's
%% automotive specialized computing system~\cite{drivepx}), which is more
%% powerful but also more expensive.

% how we trained (maybe moved to a later section)
We apply a standard imitation learning methodology to train the car to
follow tracks on the ground. We collect data for
training and validation by manually
controlling the RC car and recording the vision (from the webcam
mounted on the RC-car) and the human control inputs. We then train the
network offline using the collected data on a desktop computer, which
is equipped with a NVIDIA GTX 1060 GPU. Finally, the trained network is copied
back to the Raspberry Pi, which is then used to perform inference
operations---locally on the Pi---in the car's main control loop in
real-time. For real-time control, each inference operation must
be completed within the desired control period (e.g., 33.$\overline{\mbox{33}}$ ms
 period for 30 Hz control frequency).
% how we evaluated (in terms real-time performance)

% what are our findings?
Using the DeepPicar platform, we systematically analyze its real-time
capabilities in the context of deep-learning based real-time
control, especially on real-time deep neural network inferencing.
We also evaluated other, more powerful, embedded computing
platforms to better understand achievable real-time performance of
DeepPicar's deep-learning based control system and the impact of
computing hardware architectures.


We find that the DeepPicar is capable of 
completing control loops in under 33.$\overline{\mbox{33}}$ 
ms, or 30 hz, and can do so 100\% of the time.
Other tested embedded platforms, Intel UP and NVIDIA TX2, offer even
better performance, and are capable of supporting deep-learning based 
real-time control up to 100 Hz control frequency on the TX2 when the GPU 
is used. However, in all platforms, shared resource contention remains 
an important issue as we observe up to 11.6X control loop execution time
increase, mostly due to increase in the neural network inferencing
operation, when memory performance intensive applications are
co-scheduled on idle CPU cores.

% contributions
The {\bf contributions} of this paper are as follows:
\begin{itemize}
  \item We present the design and implementation of a
    low-cost autonomous vehicle testbed, DeepPicar, which utilizes
    state-of-the-art artificial intelligence techniques.
  \item We provide an analysis and case-study of real-time issues in the
    DeepPicar.
  \item We systematically compare real-time computing capabilities of
    multiple embedded computing platforms in the context of
    vision-based autonomous driving.
\end{itemize}

The remaining sections of the paper are as follows: Section II 
provides a background of applications in autonomous driving and related works.
 Section III gives an overview of the DeepPicar platform, including the 
high-level system and the methods used for training and inference. 
Section IV presents our evaluation of the platform and how different 
factors can affect performance. Section V gives a comparison between 
the Raspberry Pi 3 and other embedded computing platforms to 
determine their suitably for autonomous driving research. The paper finishes with 
conclusions in Section VI.

\section{Background} \label{sec:background}

\subsection{Deep End-to-End Control}
End-to-end training of deep neural network models is an active
research area in the fields of AI~\cite{Levine2016}.

\begin{figure}[h]
  \centering
  \includegraphics[width=.5\textwidth]{figs/endtoend}
  \caption{Standard robotics control vs. DNN based end-to-end
    control. \fixme{figure must be redrawn}}
\end{figure}

- explosion of AI
- in particular, application of DNN in perception and control of robotics systems.
- end-to-end control is a promising technique.
  levine's publications?
- examples: nvidia's DAVE-II prototype, forest navigating drone
challenge problem: computing at low cost?

%% UPenn's f1/10 BOM: $3,628.37	
%% http://f1tenth.org/
%% http://selfdrivingcars.mit.edu/
%% http://fast.scripts.mit.edu/racecar/
%% https://github.com/mit-racecar
%% https://mit-racecar.github.io/

\subsection{Embedded Multicore Single-Board-Computers}

- computing has been a obstacle.
- performance, but also constraints size, weight, and power as well as
cost senstive nature of industries including automtive.
- many new embeded computing platforms emerged: affordable and
powerful. raspberry pi, nvidia's embedded platforms tout their
superiority in GPU based acceleration of the AI tasks.

Our primary objective of this study are
- to understand the necessary computing performance for applying AI
technology based robotics systems, and 
- what kind of computing architecture and runtime supports
are most appropriate for such workload.

Toward to achieve the two goals, we implement a low-cost autonomous
car platform as a case study, as we will explain in the following
section. 

\section{DeepPicar Overview}

\begin{figure}[h]
  \centering
  \includegraphics[width=.4\textwidth]{figs/DeepPicar_platform}
  \caption{DeepPicar platform.}
  \label{fig:overview}
\end{figure}

In this section, we provide an overview of our DeepPicar platform. In
developing DeepPicar, one of our primary goals is to replicate the
NVIDIA's DAVE-2 system on a smaller scale with using a low cost
multicore platform, Raspberry Pi 3. Because Pi 3's computing
performance is much lower than that of the DRIVE PX platform, used in
DAVE-2, we are interested in if and how we can process
computationally expensive neural network operations in
real-time. Specifically, inferencing (forward pass processing)
operation must be completed within each control period
duration---e.g., WCET of 50ms for 20Hz control frequency---locally on
the Pi 3 platform, although training of the network (back-propagation
for weight updates) can be done offline and remotely using a desktop
computer.

Figure~\ref{fig:overview} shows the DeepPicar, which is comprised of a
set of inexpensive components: a Raspberry Pi 3 Single Board Computer
(SBC), a Pololu DRV8835 motor driver, a Playstation Eye Webcam, a
battery, and a 1:24 scale RC car. Table~\ref{tbl:carbom} shows
estimated cost of the system.

\begin{table}[h]
  \centering
  \begin{tabular}{|l|l|}
    \hline
    Item                    & cost (\$) \\
    \hline
    Raspberry Pi 3 Model B  & 35 \\
    New Bright 1:24 scale RC car       & 10 \\
    Playstation Eye camera  &  7 \\
    Pololu DRV8835 motor hat&  8 \\
    External battery pack   & 10 \\
    \hline
    Total                   & 70 \\
    \hline
  \end{tabular}
  \caption{DeepPicar's bill of materials (BOM)}
  \label{tbl:carbom}
\end{table}

For the neural network architecture, we adopt a tensorflow version of
NVIDIA's DAVE-2 convolutional neural network (CNN), published by
Fridman at  MIT~\footnote{https://github.com/lexfridman/deeptesla}. As
in NVIDIA's DAVE-2, the CNN takes raw color image (200x66 RGB pixels)
as input and produce a single steering angle value as
output. Figure~\ref{fig:architecture} shows the network architecture, which
is comprised of 9 layers, 250K parameters, and about 27 million
connections.

\begin{figure}[h]
  \centering
  \includegraphics[width=.4\textwidth]{figs/architecture}
  \caption{DeepPicar's neural network architecture: 9 layers (5
    convolutional, 4 fully-connected layers), 250K
    parameters. Adopted from ~\cite{Bojarski2016}.}
  \label{fig:architecture}
\end{figure}

% data collection and training.
To collect the training data, a human pilot manually drives the RC car
on a small track we created (Figure~\ref{fig:track}) to record
timestamped video and contol commands. The stored data is then copied 
to a desktop computer, which equips a NVIDIA GTX 1060 GPU, where we
train the network to accerate training speed. 
For comparison, training the network on the Raspbeery Pi 3 takes
approximately 4 hours, whereas it takes only about 5 minutes on the
desktop computer using the GTX 1060 GPU.

\begin{figure}[h]
  \centering
  \includegraphics[width=.4\textwidth]{figs/track_new2}
  \caption{A track for training/testing.\fixme{replace this}}
  \label{fig:track}
\end{figure}


% inferencing on pi3
Once the network is trained on the desktop computer, the trained model
is copied back to the Raspberry Pi computer. The network is then used
by the car's main controller, which feeds an image frame the web
camera as input to the network. The produced steering angle output is
then converted as the PWM value of the steering motor of the
car. Figure~\ref{fig:controlloop} shows simplified psuedo code of the
controller's main loop.

\begin{figure}[t]
   \lstset{language=python,
           basicstyle=\ttfamily\small,
           keywordstyle=\color{blue}\ttfamily,
           stringstyle=\color{red}\ttfamily,
           commentstyle=\color{green}\ttfamily
          }  
  \lstinputlisting[language=python]{control.py}
  \caption{Control loop}
  \label{fig:controlloop}
\end{figure}

\fixme{Discuss the limitation of the RC car platform: discrete
  control, web camera latency}

%% Although the steering angle output of the network is a continuous
%% real value,

%% We employ a small and relatively inexpensive RC car that is capable of
%% performing basic automotive operations. However, the car we use
%% doesn't replicate the capabilities of other autonomous vehicle
%% platforms, as it is more simplistic in nature. Notably, the RC car we
%% use only has three possible options in terms of turning: center, left,
%% and right. As such, control of the car may be less precise at times,
%% and may negatively affect performance. The car is capable of other
%% operations that aren't used within the scope of this
%% platform. Specifically, the car is able to drive in reverse and the
%% driving speed can be changed. In our experiments, we have the RC car
%% drive forward, and at a constant speed. 

%% Our platform also comes equipped with a camera that is used for both
%% recording training videos and providing input frames to the model
%% whenever the Picar is self-driving. We chose the Playstation Eye
%% Camera due to its ability to reach and maintain higher fps levels
%% while also remaining relatively low-cost. One concern, however, is the
%% affect of camera latency on the self-driving performance of our
%% platform. While humans are able to see environmental changes in
%% real-time, the same can't be said for cameras since there is a delay
%% between when frame capture by the camera, and when it is read by the
%% Raspberry Pi 3. As a result, it is possible for the model to be given
%% an input frame that is different from the real world, thus impacting
%% the performance of the Picar.

%% Compared to other existing autonomous vehicle platforms, such as the
%% F1/10 and NVIDIA DAVE-2, the Picar platform is capable of performing
%% the same operations while using more cost inexpensive components. As
%% shown by Table 1, the components selected for the Picar all cost
%% considerably less than the other  platforms when focusing on the
%% common components (camera, power supply, etc.). Including the other
%% components used in the F1/10 and DAVE-2, such as the additional
%% sensors employed by both, the difference  in cost would be even
%% greater.

%% \subsection{Model Training}
%% In order to operate the Picar as an autonomous vehicle, we utilize the
%% DeepTesla library, which is capable of training a deep neural network
%% (DNN) with end-to-end learning, that could then be used by the  Picar
%% for autonomous driving. Please note that it is more efficient to
%% execute the actual model training  process on a different system with
%% an NVIDIA GPU, rather than on the Raspberry Pi 3 itself. This is
%% because DeepTesla trains the DNN with the Tensorflow library, which
%% greatly benefits from the use of  gpu-enabled operations. For
%% comparison, training a model on the Pi took approximately 6 hours,
%% whereas  the time it took on a computer with a NVIDIA GPU was only
%% around 5 minutes!

%% We train our model on a custom made track/lane composed of black and
%% white duct tape, where the black  tape represents the bounds of the
%% track, and the white tape marks the center of the lane. The model is
%% taught to stay close to the center of the lane for as long as
%% possible, and to turn whenever it  reaches/crosses the outer lane
%% bounds so that it remains on the track. For data, we navigate and
%% record  the car going both ways across the track, and use the
%% collected videos to train the model(s) that will  be used later for
%% angle prediction while the car is self-driving.

%% \begin{table*}[h]
%%   \centering
%%   \begin{tabular} {| l | l | l | l | l |}
%%     \hline
%%     \textbf{Platform} & \textbf{Picar} & \textbf{F1/10} & \textbf{NVIDIA DAVE-2}\\ \hline  
%%     Car & Mini RC Car (\$10) & Traxxas 1/10th car platform (\$299.97) & TBD\\ \hline
%%     Embedded system & Raspberry Pi 3 Model B (\$35.00) & NVIDIA Jetson TK1 (\$192.99) & NVIDIA DRIVE PX \\ \hline
%%     CPU & Cortex A-53 quad core & Cortex A-15 quad core & 8x Cortex A-57 quad core, 8x Cortex A-53 quad core \\ \hline
%%     Camera & Playstation Eye camera (\$6.99) & ZED camera (\$449.00) & 3x cameras\\ \hline
%%     Power Supply & Mobile battery (\$5.99) & Energizer battery pack (\$159.00) & TBD\\
%%     \hline
%%   \end{tabular}
%%   \caption{Comparison of components used in autonomous vehicle platforms.}
%% \end{table*}



\section{Platform Overview}

\begin{figure}[h]
  \centering
  \includegraphics[width=.4\textwidth]{Picar_Picture}
  \caption{ The Picar Platform. }
\end{figure}

\subsection{Platform Components}
We employ a small and relatively inexpensive RC car that is capable of performing basic automotive 
operations. However, the car we use doesn't replicate the capabilities of other autonomous vehicle 
platforms, as it is more simplistic in nature. Notably, the RC car we use only has three possible 
options in terms of turning: center, left, and right. As such, control of the car may be less precise at 
times, and may negatively affect performance. The car is capable of other operations that aren't used 
within the scope of this platform. Specifically, the car is able to drive in reverse and the driving 
speed can be changed. In our experiments, we have the RC car drive forward, and at a constant speed.

Our platform also comes equipped with a camera that is used for both recording training videos and 
providing input frames to the model whenever the Picar is self-driving. We chose the Playstation Eye 
Camera due to its ability to reach and maintain higher fps levels while also remaining relatively 
low-cost. One concern, however, is the affect of camera latency on the self-driving performance of our 
platform. While humans are able to see environmental changes in real-time, the same can't be said for 
cameras since there is a delay between when frame capture by the camera, and when it is read by the 
Raspberry Pi 3. As a result, it is possible for the model to be given an input frame that is different 
from the real world, thus impacting the performance of the Picar.

\subsection{Model Training}
In order to operate the Picar as an autonomous vehicle, we utilize the DeepTesla library, which is 
capable of training a deep neural network (DNN) with end-to-end learning, that could then be used by the 
Picar for autonomous driving. Please note that it is more efficient to execute the actual model training 
process on a different system with an NVIDIA GPU, rather than on the Raspberry Pi 3 itself. This is 
because DeepTesla trains the DNN with the Tensorflow library, which greatly benefits from the use of 
gpu-enabled operations. For comparison, training a model on the Pi took approximately 6 hours, whereas 
the time it took on a computer with a NVIDIA GPU was only around 5 minutes! 

We train our model on a custom made track/lane composed of black and white duct tape, where the black 
tape represents the bounds of the track, and the white tape marks the center of the lane. The model is 
taught to stay close to the center of the lane for as long as possible, and to turn whenever it 
reaches/crosses the outer lane bounds so that it remains on the track. For data, we navigate and record 
the car going both ways across the track, and use the collected videos to train the model(s) that will 
be used later for angle prediction while the car is self-driving.

\section{Data Collection}
...
\section{Network Architecture}
...
\section{Real-Time DNN Inferencing}
...
\section{Online/Offline Training}
...
\section{Evaluation}
In evaluating the real-time efficacy of the Raspberry Pi 3, the same methodology is utilized in all 
experiments conducted. The performance of the Pi is measured over a set of 1001 video frames that are 
each individually fed to the model. The processing time for the first frame is ignored as, due to 
cache warmup, it is uncharacteristically high and doesn't accurately represent the Pi's capabilities. A 
deadline of 50 ms, or 20 Hz, is used as a baseline to assess the Pi's ability to complete all 
necessary real-time operations in a timely manner. Also, please note that frame processing times 
would be approximately the same if the input stream was a 
camera instead of a video. 

\subsection{Real-Time Operations}
In a real-time operations, the system has to be capable of consistently executing all necessary 
functions before their given deadlines. In the case of our platform, it has to process every given frame 
and get the predicted angles from the model. In order to determine if the Raspberry Pi 3 is capable of 
this, we tested the platform and recorded the time it took for the Pi to complete all real-time 
functions for every given frame.For this experiment, all four of the Pi's cpu cores were utilized, and 
only one model was run. As can be seen in Figure 1, the Pi was able to meet almost all of its deadlines, 
while only missing it in a few instances.

\begin{figure}[h]
  \centering
  \includegraphics[width=.5\textwidth]{Total_Processing_Time}
  \caption{ Real-time performance of the Raspberry Pi 3 in processing 1000 frames.}
\end{figure}

In our platform, three main real-time tasks are performed during autonomous operation. 
In order, those operations are: (1) capturing and reading the input frame from the designated camera 
or video stream, (2) preprocessing the acquired frame so that it is compatible with the DNN, and (3) 
feeding the frame to, and getting the angle prediction from, the model. However, the time it takes to 
complete each of the operations will differ, with at least one of them being the dominating step in 
processing each frame.

\begin{table}[h]
  \centering
  \begin{tabular} {| l | l | l | l | l |}
    \hline
    \textbf{operation} & \textbf{mean} & \textbf{max} & \textbf{99pct} & \textbf{stdev} \\ \hline 
    Frame Capture & 2.28 & 4.94 & 63.31 &  .51\\ \hline
    Preprocessing & 3.09 & 4.6 & 3.31 & .1 \\ \hline
    Angle Prediction & 37.3 & 51.03 & 45.48 & 2.75 \\ 
    \hline
  \end{tabular}
  \caption{Real-time performance of the Raspberry Pi 3 depending on the number of cores used.}
\end{table}

In order to determine which operation(s) take the longest to execute, we measured the time it 
took for each step to be completed. For this experiment, all four of the Pi's cpu cores were utilized, 
and only one model was run. As is shown in Fig. 1, the angle prediction operation (3) consumes the 
majority of the processing for each frame. Furthermore, the time it takes for the operation to 
complete is volatile, and can range anywhere between 30 ms and 50 ms for any particular frame. On the 
other hand, both the frame capture (1) and preprocessing (2) operations take substantially less time 
and are relatively more consistent in their times, at 2 ms and 3 ms, respectively. 

\subsection{Multicore Performance}
It may not always be the case that all four cores of the Raspberry Pi 3's Cortex A-53 CPU can be used 
solely for the purpose of operating an autonomous vehicle. Thus, we test how the number of cores 
utilized for real-time operations affects the Pi's overall ability to function as an autonomous 
vehicle platform.

\begin{table}[h]
  \centering
  \begin{tabular} {| l | l | l | l | l | l | l | l | l | l |}
    \hline
    \textbf{num cores} & \textbf{mean} & \textbf{max} & \textbf{99pct} & \textbf{stdev} \\ \hline 
    1 & 61.96 & 66 & 63.31 &  .51\\ \hline
    2 & 50.49 & 71.55 & 70.03 & 4.18 \\ \hline
    3 & 48.11 & 72.22 & 58.45 & 2.8 \\ \hline
    4 & 42.67 & 56.37 & 50.70 & 2.8 \\
    \hline
  \end{tabular}
  \caption{Real-time performance of the Raspberry Pi 3 depending on the number of cores used.}
\end{table}

As is depicted in Table 1, the Raspberry Pi performed better, on average, when it utilized more 
cores. With 4 cores, the Pi was able to meet the vast majority of its 50 ms deadlines, doing so in 
almost 99\% of the processed frames. The Pi had the worst average performance when using only 1 core, 
as it was unable to meet any of the 50 ms deadlines (its fastest processing time was just under 60 
ms). Another observation is that the difference between using 2 cores and 3 cores is relatively 
small. On average, using 3 cores only performed better by 2 ms, so the addition of one core in that 
specific case offers relatively little improvement. However, please note that the average time when 
using 3 cores was below the 50 ms deadline, while the average time using 2 cores was slightly above it. 
One important observation that can be made is that of consistency, as using only 1 core had a much 
steadier performance. As a result, the use of multiple cores is very beneficial in terms of reducing 
the time it takes to complete real-time operations, but may ultimately result in processing times 
that are more volatile.

\subsection{Multimodel Performance}
In a real world scenario, it is highly probable that multiple models will need to be used 
simultaneously in order to perform various important tasks (angle prediction, object detection, 
etc.)\cite{}. As such, we also tested the capability of the Raspberry Pi to run multple models at the 
same time, and measure whether all models are able to meet their respective deadlines on a consistent 
basis. Specifically, the Pi is tested in the cases of running 2 and 4 models simultaneously. For each 
case, all models are allocated an equal number of cores, with 2 models having 2 cores each, and 4 
models having 1 core each.

\begin{table*}
  \begin{tabular} {| l | l | l | l | l | l | l | l | l |}
  \hline
  \textbf{num models} & \textbf{cores} & \textbf{mean} & \textbf{L1 refs} & \textbf{L1 
    misses} & \textbf{L1 miss \%} & textbf{L2 refs} & \textbf{L2 misses} & \textbf{L2 miss \%} \\ \hline
  1 & 0,1 & 51.35 & 3.04E+10 & 4.78E+08 & 1.58 & 3.31E+09 & 3.68E+08 & 11.12\\ \hline
  2 & 0,1 & 58.03 & 3.04E+10 & 4.91E+08 & 1.61 & 3.91E+09 & 4.26E+08 & 10.88 \\ \hline
  2 & 2,3 & 56.4 & 3.04E+10 & 4.80E+08 & 1.58 & 3.88E+09 & 4.21E+08 & 10.87 \\ \hline
  \end{tabular}
  \caption{Real-time performance of the Raspberry Pi 3 when 2 models are running simultaneously.}
\end{table*}

Ideally, each model run would replicate a single model run. However, Table 2 shows that such was not 
the case. In the tests where 2 models ran simultaneously, both of the models showed average time 
increases of around 5-7 ms, around 10\%, when compared to a baseline of 1 model running on two cores. 
This change, however, did not seem to be caused by any form of cache interference as both L1 and L2 
cache misses remained constant, regardless of the number of models running. The number of L1 misses 
being close to 1.6\% of all references and the number of L2 misses being about 11\% of all 
references.

\begin{table*}
  \begin{tabular} {| l | l | l | l | l | l | l | l | l |}
  \hline
  \textbf{num models} & \textbf{core} & \textbf{mean} & \textbf{L1 refs} & \textbf{L1 
    misses} & \textbf{L1 miss \%} & textbf{L2 refs} & \textbf{L2 misses} & \textbf{L2 miss \%} \\ \hline
    1 & 0 &  62.48 & 2.78E+10 & 4.36E+08 & 1.57 & 2.83E+09 & 3.59E+08 & 12.68  \\ \hline
    4 & 0 & 77.9 & 2.79E+10 & 4.53E+08 & 1.63 & 3.36E+09 & 4.43E+08 & 13.19 \\ \hline
    4 & 1 & 78.89 & 2.79E+10 & 4.64E+08 & 1.67 & 3.42E+09 & 4.38E+08 & 12.82 \\ \hline
    4 & 2 & 77.81 & 2.78E+10 & 4.45E+08 & 1.6 & 3.45E+09 & 4.41E+08 & 12.77 \\ \hline
    4 & 3 & 77.87 & 2.79E+10 & 4.45E+08 & 1.60 & 3.41E+09 & 4.39E+08 & 12.88 \\ \hline
  \end{tabular}
  \caption{Real-time performance of the Raspberry Pi when 4 models are running simultaneously.}
\end{table*}

The difference was even greater in the case of 4 models running concurrently, as each one displayed 
an average time increase of approximately 15 ms, around 30\%, when compared to a single model 
running on 1 core. Once again, though, cache interference was not a factor as the number of L1 cache 
misses remained 1.6\% of all references, and the number of L2 cache misses stayed at 13\% of all 
references.

\subsection{Performance Requirements}
In the utilization of the Raspberry Pi 3 in our platform, there are a few factors that need to be 
considered and/or enforced in order to guarantee that the Pi is able to consistently perform at a 
desired level. Specifically, these issues all have the potential to negatively affect the cpu clock 
speed/frequency, which would result in decreased performance. While, in the above experiments, the cpu 
operated at a preferred clock speed of 1.2 GHz, it is entirely possible for the cpu to operate at a 
lower frequency if the following problems are not taken into account.

The most notable issue that can affect the cpu clock speed is that of the power supplied to the 
Raspberry Pi. In essence, it is necessary that the Pi be supplied with 2 Amps, as any less could 
hinder the Pi's ability to maintain a 1.2 GHz frequency. In experiments done with a power supply that 
only provided 1 Amp, the Pi was unable to sustain a 1.2 GHz clock speed and, instead, fluctuated 
between operating at 600 MHz and 1.2 GHz. As a result, it is necessary, or at least highly 
recommended, that the power supply used for the Raspberry Pi 3 be capable of outputting 2 Amps, 
otherwise optimal performance isn't guaranteed.

Another factor that can affect clock speed is that of the cpu's temperature. Some model operations can 
be computationally intensive, thus it is possible for the temperature of the cpu to become relatively 
high. This can be especially problematic in situations where multiple models are running 
simultaneously on the Pi. As a consequence, thermal throttling may be used to decrease the clock 
speed so that the cpu temperature stays at a safe level. As such, the Raspberry Pi may not be suited 
for prolonged use, especially in cases where the workload is relatively larger, like running multiple 
models. Rather, the Pi seems to be better suited for running in set periods, after which it is turned 
off or made idle so that the cpu is given time to cool down.
\section{Related Work}
\cite{Bojarski2016}

\section{Conclusion}
%-------------------------------------------------------------------------

\bibliographystyle{plain}
\bibliography{reference}
\end{document}
